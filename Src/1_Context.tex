% *developing country -> average walk of 6km a day to get water. Conventional motivation for renewable energy does not apply. 

% *little access to electricity 38.2\% in 2021.

% *7TWh of electricity consumed in 2020 -> 0.1mWh per capita. compared to uk 4.6mWh per capita 

% * CO2 intensity of energy mix 11.9t/Tj compared to 48.5 t/Tj in the UK. Primary driving factor for renewable is to reduce CO2 emissions and hence reduce global warming.  



Tanzania is considered a developing country. Its GDP per capita as of 2021 is \$1099 compared to \$46,510 in the UK, ranking it one of the poorest countries in the world \cite{worldBank}. The average Tanzanian walks 6km a day to get water. Conventional motivations for renewable energy, reducing emotions to address climate change, do not apply!

Despite this, key RE indicators suggest that renewable energy is prevalent in Tanzania compared to other countries. For example, the CO2 intensity of the energy mix is 11.9t/Tj compared to 48.5 t/Tj in the UK \cite{iea}.  

Tanzania is a country that is still discovering electricity. As of 2021, only 38.2\% of the population has access to electricity. The average American uses 127 times more electricity than the average Tanzanian. The proportion of electricity that makes up the net energy consumption is almost insignificant; it is so tiny \cite{iea}.



