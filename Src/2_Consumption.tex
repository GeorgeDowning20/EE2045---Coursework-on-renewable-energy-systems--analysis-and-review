% * the vast majority of energy consumption comes from the burning of biomass.

%     * about 1/3rd of electricity is generated by hydroelectric power plants. the


%     * Tanzania generally has a comfortable climate all year round averaging 26.55 degrees celsius with minimal variation. Natural temperature regulating systems are used. a primary use of energy consumption in first world country's.


% Tanzania 
 
% Tanzania sources 90 percent of its energy needs / 

% * breakdown of energy per capita compare to UK/America. net energy, electricity

% Review why energy consumption is so low. -- low gdp, smal industrial economy ->  average person can't afford. comfortable living climate. poor infrastructure for electricity. luxury such as cars that are typically energy intensive are less commen, linking to the price.


Tanzania's yearly energy consumption per capita is 16.8GJ, significantly less than other countries, such as 266.3GJ in America and 102.2GJ in the UK. Of the 16.8GJ the average Tanzanian consumes, only 0.36GJ or 0.1GWh is electricity. This is 2.1\% of the total energy consumption, compared to America at 17.2\% and the UK at 16.5\%. 0.1GWh is equivalent to running a single 60W lightbulb for 5 hours a day for a year \cite{iea}. 

\begin{table}[ht!]
    \centering
    \begin{tabular}{|c|c|c|c|}
    \hline
                                                                                                                        & Tanzania & United Kingdom & United States \\ \hline
    \begin{tabular}[c]{@{}c@{}}Average Energy Consumption\\  {[}GJ/Year/Capita{]}\end{tabular}                          & 16.8     & 102.2          & 266.3         \\ \hline
    \begin{tabular}[c]{@{}c@{}}Average Electricity Consumption\\  {[}GWh/Year/Capita{]}\end{tabular}                    & 0.1      & 4.6            & 12.7          \\ \hline
    \begin{tabular}[c]{@{}c@{}}Average Energy Consumption over\\  Average Electricity Consumption {[}\%{]}\end{tabular} & 2.1      & 16.5           & 17.2          \\ \hline
    \end{tabular}
    \end{table}

    There are several contributing reasons for the small average energy consumption of Tanzania. The most obvious reason is the low GDP per capita of Tanzania. The average person in Tanzania cannot afford to consume excess energy. Two major energy sinks of first-world countries are the transport and industrial sectors; these are less prevalent due to the country's developing status, reducing the average energy consumption of the country. Additionally, the climate of Tanzania is comfortable all year round, with minimal variation, allowing natural temperature regulating systems to be used, eliminating a primary use of energy consumption in first-world countries. 

    Tanzania's tiny yearly electrical energy consumption, 0.1GWh, can be attributed to the country's developing status. The infrastructure for electricity is abysmal, with only 38.2\% of the population having access to electricity. While electricity is cheaper than other energy sources, implementing the infrastructure and electrical appliances is too expensive for most of the population.




% Thirdly, the infrastructure for electricity is poor. The majority of the population does not have access to electricity, and the infrastructure is limited. Finally, luxuries such as cars that are typically energy intensive are less common, linking to the price. 

% Given 38.2 percent of the population have access to electricity, this implies there is a skew in the statistic, where the majority of the population doesn't use any electricity, and the minority uses a lot.


% Where the grid is limited or non-existent, portable renewable energy systems are needed.

